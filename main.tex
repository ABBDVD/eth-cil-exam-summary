\documentclass[a4paper, 11pt, landscape]{article}

\usepackage{mathptmx} % more compact font
\usepackage{amsmath}
\usepackage{amsfonts}
\usepackage{mathtools}
\usepackage{multicol}
\usepackage{enumitem}
\usepackage{paralist} % for compacter lists
\usepackage{hyperref} % for Todo's and similar things
\usepackage[margin=0.5cm, landscape, nohead, nofoot]{geometry}
\usepackage[small,compact]{titlesec}
\usepackage[usenames,dvipsnames,svgnames,table]{xcolor}
\usepackage{xparse}

% compact text
\linespread{0.9}
\setlength{\parindent}{0pt}

% compact lists even more
\setdefaultleftmargin{0em}{0em}{0em}{0em}{0em}{0em}

% compact sections
\titlespacing*{\section}{0pt}{0em}{0em}
\titlespacing*{\subsection}{0pt}{0em}{0em}
\titlespacing*{\subsubsection}{0pt}{0em}{0em}

% coloured section headings for easier read
\titleformat{name=\section}[block]
  {\sffamily}
  {}
  {0pt}
  {\colorsection}
\newcommand{\colorsection}[1]{%
	\colorbox{red!10}{\parbox[t][0em]{\dimexpr\columnwidth-2\fboxsep}{\thesection\ #1}}}


\titleformat{name=\subsection}[block]
  {\sffamily}
  {}
  {0pt}
  {\subcolorsection}
\newcommand{\subcolorsection}[1]{%
	\colorbox{orange!10}{\parbox[t][0em]{\dimexpr\columnwidth-2\fboxsep}{\thesubsection\ #1}}}


\titleformat{name=\subsubsection}[block]
  {\sffamily}
  {}
  {0pt}
  {\subsubcolorsection}
\newcommand{\subsubcolorsection}[1]{%
	\colorbox{blue!10}{\parbox[t][0em]{\dimexpr\columnwidth-2\fboxsep}{\thesubsubsection\ #1}}}
	
% environment for multicols inside a list
\NewDocumentEnvironment{listcols}{O{2} O{0pt}}
	{%
		\bgroup %
		\setlength{\multicolsep}{#2} %
		\begin{multicols*}{#1} %
	}
	{%
		\end{multicols*} %
		\egroup %
	}

% multicols lines & spacing
\setlength{\columnsep}{0.2cm}
\setlength{\columnseprule}{0.2pt}

% No page numbers
\pagenumbering{gobble}

% math helpers
\DeclareMathOperator*{\argmin}{arg\,min}
\DeclareMathOperator*{\argmax}{arg\,max}

\begin{document}
\begin{multicols*}{3}

\section{PCA}
$\mathbf{X} \in \mathbb{R}^{D \times N}$. $N$ observations, $K$ properties. Target: $\tilde{\mathbf{X}} \in \mathbb{R}^{K \times N}$.
\begin{compactenum}
	\item Empirical Mean: $\overline{\mathbf{x}} = \frac{1}{N} \sum_{n=1}^N \mathbf{x}_n$
	\item Center Data: $\overline{\mathbf{X}} = \mathbf{X} - [\overline{\mathbf{x}}, \ldots, \overline{\mathbf{x}}] = \mathbf{X} - \mathbf{M}$
	\item Covariance Matrix: $\boldsymbol{\Sigma} = \frac{1}{N	} \sum_{n=1}^N (\mathbf{x}_n - \overline{\mathbf{x}}) (\mathbf{x}_n - \overline{\mathbf{x}})^T = \frac{1}{N} \overline{\mathbf{X}}\overline{\mathbf{X}}^T$
	\item Eigenvalue Decomposition: $\boldsymbol{\Sigma} = \mathbf{U} \boldsymbol{\Lambda} \mathbf{U}^T$, sort eigenvalues (and eigenvectors) in descending order
	\item Select $K < D$, keep only the first $K$ eigenvalues and corresponding eigenvectors $\Rightarrow \mathbf{U}_K, \boldsymbol{\lambda}_K$
	\item Transform data onto new Basis: $\overline{\mathbf{Z}} = \mathbf{U}_K^T \overline{\mathbf{X}}$
	\item Reconstruct to original Basis: $\tilde{\overline{\mathbf{X}}} = \mathbf{U}_k \overline{\mathbf{Z}}$
	\item Reverse centring: $\tilde{\mathbf{X}} = \tilde{\overline{\mathbf{X}}} + \mathbf{M}$
\end{compactenum}

\begin{compactitem}
	\item For compression save $\mathbf{U}_k, \overline{\mathbf{Z}}, \overline{\mathbf{x}}$.
	\item $\mathbf{U}_k \in \mathbb{R}^{D \times K}, \boldsymbol{\Sigma} \in \mathbb{R}^{D \times D}, \overline{\mathbf{Z}} \in \mathbb{R}^{K \times N}, \overline{\mathbf{X}} \in \mathbb{R}^{D \times N}$
\end{compactitem}

\begin{compactdesc}
	\item[Approx. Error:] Sum of discarded eigenvalues
\end{compactdesc}

\section{Eigenvalue / -vectors}
Eigenvalue Problem: $\mathbf{Ax} = \lambda \mathbf{x}$
\begin{compactenum}
	\item solve $\operatorname{det}(\mathbf{A} - \lambda \mathbf{I}) \overset{!}{=} 0$ resulting in $\{\lambda_i\}_i$
	\item $\forall \lambda_i$:
		solve $(\mathbf{A} - \lambda_i \mathbf{I}) \mathbf{x}_i = \mathbf{0}$, $\mathbf{x}_i$ is the $i$-th eigenvector.
	\item (opt.) normalize eigenvector $q_i$: $q_i^{\text{norm}} = \frac{1}{\|q_i\|_2} q_i$.
\end{compactenum}

\section{Eigendecomposition}
$\mathbf{A} \in \mathbb{R}^{N \times N}$ with $N$ linear independent eigenvectors, then: $\mathbf{A} = \mathbf{Q} \boldsymbol{\Lambda} \mathbf{Q}^{-1}$. $\mathbf{Q} \in \mathbb{R}^{N \times N}$, $i$-th column is the $i$-th eigenvector of $\mathbf{A}$. $\boldsymbol{\Lambda}$ is diagonal and $\boldsymbol{\lambda}_{i,i}$ is the $i$-th eigenvector. Every symmetric matrix can be ``eigendecomposed''. 
$\mathbf{A}^{-1} = \mathbf{Q} \boldsymbol{\Lambda}^{-1} \mathbf{Q}^{-1}$, $(\boldsymbol{\Lambda}^{-1})_{i,i} = \frac{1}{\lambda_i}$
Calculate eigenvalues $\lambda_i$ and eigenvectors $q_i$. Normalise the eigenvectors. Put eigenvectors as columns into $\mathbf{Q}$ and eigenvalues into the diagonal of $\boldsymbol{\Lambda}$.

\section{SVD}
\begin{compactitem}
	\item $\mathbf{A} = \mathbf{U} \mathbf{D} \mathbf{V}^T = \sum_{k=1}^{\operatorname{rank}(\mathbf{A})} d_{k,k} u_k (v_k)^T$
	\item $\mathbf{A} \in \mathbb{R}^{N \times P}, \mathbf{U} \in \mathbb{R}^{N \times N}, \mathbf{D} \in \mathbb{R}^{N \times P}, \mathbf{V} \in \mathbb{R}^{P \times P}$
	\item $\mathbf{U}^T \mathbf{U} = I = \mathbf{V}^T \mathbf{V}$ ($\mathbf{U}, \mathbf{V}$ columns are orthonormal)
	\item $\mathbf{U}$ columns are eigenvectors of $\mathbf{A} \mathbf{A}^T$, $\mathbf{V}$ columns are eigenvectors of $\mathbf{A}^T \mathbf{A}$, $\mathbf{D}$ diagonal elements are singular values, i.e. the square roots of the eigenvalues ($\mathbf{A}^T \mathbf{A}$ and $\mathbf{A} \mathbf{A}^T$ have the same eigenvalues)
	\item $(\mathbf{D}^{-1})_{i,i} = \frac{1}{\mathbf{D}_{i, i}}$ ($\mathbf{D} \in \mathbb{R}^{N \times P} \to \mathbf{D}^{-1} \in \mathbb{R}^{P \times N}$, i.e. don't forget to transpose)
	\item Missing columns in $\mathbf{U}$ are basis of $\operatorname{null}(A^T)$ and in $\mathbf{V}$ are basis of $\operatorname{null}(A)$. Calculate: $\mathbf{A}^T \mathbf{u} = \mathbf{0}$ or $\mathbf{A} \mathbf{v} = \mathbf{0}$ for $\mathbf{u}$ or $\mathbf{v}$.
\end{compactitem}

\begin{compactenum}
	\item calculate $\mathbf{A}^T \mathbf{A}$.
	\item calculate eigenvalues of $\mathbf{A}^T \mathbf{A}$, the square root of them are the diagonal elements of $\mathbf{D}$.
	\item calculate eigenvectors of $\mathbf{A}^T \mathbf{A}$ using the eigenvalues resulting in the columns of $\mathbf{V}$.
	\item calculate the missing matrix: $\mathbf{U} = \mathbf{A} \mathbf{V} \mathbf{D}^{-1}$. Can be checked by calculating the eigenvectors of $\mathbf{A} \mathbf{A}^T$.
	\item normalize each columns of $\mathbf{U}$ and $\mathbf{V}$.
\end{compactenum}

\subsection{Low-Rank approximation}
Using only $K$ largest eigenvalues and corresponding eigenvectors. $\tilde{\mathbf{A}}_{i, j} = \sum_{k}^K \mathbf{U}_{i, k} \mathbf{D}_{k,k} \mathbf{V}_{j, k} = \mathbf{U}_{i, k} \mathbf{D}_{k,k} (\mathbf{V}^T)_{k, j}$.
\begin{compactdesc}
	\item[Error Frobenius:] $\|\mathbf{A} - \tilde{\mathbf{A}}\|_F = \sqrt{\sum_{i > K} \sigma_i^2} = \sqrt{\sum_{i > K} \lambda_i}$
	\item[Error Euclidean:] $\|\mathbf{A} - \tilde{\mathbf{A}}\|_2 = \sigma_{K+1}$
\end{compactdesc}

\section{Basis Transform}
$\mathbf{V}, \mathbf{W}$ are two basis matrices (each column is a basis vector, invertible). Representing $\mathbf{x}$ in basis $\mathbf{V}$ by \emph{coordinate vector} $\mathbf{a}$: $\mathbf{x} = \mathbf{V} \mathbf{a} \Rightarrow \mathbf{a} = \mathbf{V}^{-1} \mathbf{x}$. Knowing $\mathbf{a}$ representing $\mathbf{x}$ in $\mathbf{V}$ we can calculate coordinate vector $\mathbf{b}$ representing $\mathbf{x}$ in $\mathbf{W}$ with: $\mathbf{x} = \mathbf{W}^{-1} \mathbf{b} = \mathbf{V}^{-1} \mathbf{a} \Rightarrow \mathbf{b} = \mathbf{W} \mathbf{V}^{-1} \mathbf{a}$.

\section{Matrix/Vector}
\begin{compactdesc}
	\item[Orthogonal:] (i.e. columns are orthonormal!) $\mathbf{A}^{-1} = \mathbf{A}^T$, $\mathbf{A} \mathbf{A}^T = \mathbf{A}^T \mathbf{A} = \mathbf{I}$, $\operatorname{det}(\mathbf{A}) \in \{+1, -1\}$, $\operatorname{det}(\mathbf{A}^T \mathbf{A}) = 1$
	\item[Inner Product:] (in $\mathbb{R}^D$) $\langle \mathbf{x}, \mathbf{y} \rangle = \mathbf{x}^T \mathbf{y} = \sum_{i=1}^{N} \mathbf{x}_i \mathbf{y}_i$.
	\begin{inparaitem}[\color{red}\textbullet]
		\item $\langle \mathbf{x} \pm \mathbf{y}, \mathbf{x} \pm \mathbf{y} \rangle = \langle \mathbf{x}, \mathbf{x} \rangle \pm 2 \langle \mathbf{x}, \mathbf{y} \rangle + \langle \mathbf{y}, \mathbf{y} \rangle$
		\item $\langle \mathbf{x}, \mathbf{y} + \mathbf{z} \rangle = \langle \mathbf{x}, \mathbf{y} \rangle + \langle \mathbf{x}, \mathbf{z} \rangle$
		\item $\langle \mathbf{x} + \mathbf{y}, \mathbf{z} \rangle = \langle \mathbf{x}, \mathbf{z} \rangle + \langle \mathbf{y}, \mathbf{z} \rangle$
		\item $\langle \mathbf{x}, \mathbf{y} \rangle = |\mathbf{x}| \cdot |\mathbf{y}| \cdot \cos(\theta)$
		\item If $\mathbf{y}$ is a unit vector then $\langle \mathbf{x}, \mathbf{y} \rangle$ projects $\mathbf{x}$ onto $\mathbf{y}$
	\end{inparaitem}
	\item[Outer Product:] $\mathbf{u} \mathbf{v}^T$, $(\mathbf{u} \mathbf{v}^T)_{i, j} = \mathbf{u}_i \mathbf{v}_j$
	\item[Transpose:] $(\mathbf{A}^T)^{-1} = (\mathbf{A}^{-1})^T$
	\item[Gram-Schmidt:] $\{\mathbf{w}_i\}_i$ non-orthogonal basis. $\mathbf{v}_n = \mathbf{w}_n - \sum_{i=1}^{n-1} \frac{\langle \mathbf{v}_i, \mathbf{w}_n \rangle}{\langle \mathbf{v}_i, \mathbf{v}_i \rangle} \mathbf{v}_i$ results in $\{\mathbf{v}_i\}_i$ an orthogonal basis (same space).
\end{compactdesc}

\section{Norms}
\begin{inparadesc}
	\item[$l_0$:] $\|\mathbf{x}\|_0 := |\{i | x_i \neq 0\}|$
	\item[Nuclear:] $\|\mathbf{X}\|_\star = \sum_{i=1}^{\min(m, n)} \sigma_i$
	\item[Euclidean:] $\|\mathbf{x}\|_2 := \sqrt{\sum_{i=1}^{N} \mathbf{x}_i^2} = \sqrt{\mathbf{x}^T \mathbf{x}} = \sqrt{\langle \mathbf{x}, \mathbf{x} \rangle}$
	\item[$p$-norm:] $\|\mathbf{x}\|_p := \left( \sum_{i=1}^{N} |x_i|^p \right)^{\frac{1}{p}}$
	\item[Frobenius:] $\|\mathbf{A}\|_F :=\allowbreak \sqrt{\sum_{i=1}^{M} \sum_{j=1}^{N} |\mathbf{A}_{i, j}|^2} =\allowbreak \sqrt{\operatorname{trace}(\mathbf{A}^T \mathbf{A})} =\allowbreak \sqrt{\sum_{i=1}^{\min\{m, n\}} \sigma_i^2}$ ($\sigma_i$ is the $i$-th singularvalue), $\mathbf{A} \in \mathbb{R}^{M \times N}$
\end{inparadesc}

\section{$K$-means Algorithm}
\begin{inparaenum}[\color{red} 1.]
	\item Initiate: choose randomly $K$ centroids $\mathbf{U} = [\mathbf{u}_1, \ldots, \mathbf{u}_K]$ (usually $K$ randomly chosen data points).
	\item Assign data points to clusters. $k^\star(\mathbf{x}_n) = \argmin_k \{ \|\mathbf{x}_n - \mathbf{u}_1\|_2^2, \ldots, \|\mathbf{x}_n - \mathbf{u}_K\|_2^2 \}$ gives us the cluster $k$ whose centroid $\mathbf{u}_k$ is closest to data point $\mathbf{x}_n$. Set accordingly the $k$-th element of the $n$-th column of the assignment matrix $\mathbf{Z}$ to $1$, others to $0$.
	\item Update centroids: $\mathbf{u}_k = \frac{\sum_{n=1}^N z_{k,n} \mathbf{x}_n}{\sum_{n=1}^N z_{k,n}}$ (each centroid is in the middle of all data points assigned to it's cluster).
	\item Repeat from step 2, stops if $\frac{1}{2} \|\mathbf{Z} - \mathbf{Z}^\text{new}\|_0 = 0$.
\end{inparaenum}

\section{Probability / Statistics}
\begin{compactitem}
	\item $P(x) := Pr[X = x] := \sum_{y \in Y} P(x, y)$
	\item $P(x|y) := Pr[X = x | Y = y] := \frac{P(x,y)}{P(y)},\quad \text{if } P(y) > 0$
	\item $\forall y \in Y: \sum_{x \in X} P(x|y) = 1$ (property for any fixed $y$)
	\item $P(x, y) = P(x|y) P(y)$
	\item $P(x|y) = \frac{P(y|x)P(x)}{P(y)}$ (Bayes' rule)
	\item $P(x|y) = P(x) \Leftrightarrow P(y|x) = P(y)$ (iff $X$, $Y$ independent)
	\item $P(x_1, \ldots, x_n) = \prod_{i=1}^n P(x_i)$ (iff IID)
\end{compactitem}

\section{Gaussian Mixture Models (GMM)}
\begin{compactdesc}
	\item[GMM:] $p(\mathbf{x}) = \sum_{k=1}^K \pi_k \mathcal{N}(\mathbf{x} | \boldsymbol{\mu}_k, \Sigma_k)$
	\item[Assignment variable:] $\mathbf{z}_k \in \{0, 1\}$, $\sum_{k=1}^K \mathbf{z}_k = 1$, $\operatorname{Pr}(\mathbf{z}_k = 1) = \boldsymbol{\pi}_k \Leftrightarrow p(\mathbf{z}) = \prod_{k=1}^K \boldsymbol{\pi}_k^{\mathbf{z}_k}$
	\item[Complete data distribution:] $p(\mathbf{x}, \mathbf{z}) = \prod_{k=1}^K \left( \boldsymbol{\pi}_k \mathcal{N}(\boldsymbol{\mu}_k, \boldsymbol{\Sigma}_k) \right)^{\mathbf{z}_k}$
	\item[Posterior Probabilities:] $\operatorname{Pr}(\mathbf{z}_k = 1 | \mathbf{x}) = \frac{\operatorname{Pr}(\mathbf{z}_k = 1) p(\mathbf{x} | \mathbf{z}_k = 1)}{\sum_{l=1}^K \operatorname{Pr}(\mathbf{z}_l = 1) p(\mathbf{x} | \mathbf{z}_l = 1)} = \frac{\boldsymbol{\pi}_k \mathcal{N}(\mathbf{x} | \boldsymbol{\mu}_k, \boldsymbol{\Sigma}_k)}{\sum_{l=1}^K \boldsymbol{\pi}_l \mathcal{N}(\mathbf{x} | \boldsymbol{\mu}_l, \boldsymbol{\Sigma}_l)}$
	\item[Likelihood of observed data $\mathbf{X}$:] $p(\mathbf{X} | \boldsymbol{\pi}, \boldsymbol{\mu}, \boldsymbol{\Sigma}) = \prod_{n=1}^N p(\mathbf{x}_n) = \prod_{n=1}^N \sum_{k=1}^K \pi_k \mathcal{N}(\mathbf{x}_n | \boldsymbol{\mu}_k, \boldsymbol{\Sigma}_k)$
	\item[Log-likelihood:] $\ln p(\mathbf{X} | \boldsymbol{\pi}, \boldsymbol{\mu}, \boldsymbol{\Sigma}) =\break \sum_{i=1}^N \ln \left( \sum_{k=1}^K \pi_k \mathcal{N}(\mathbf{x}_i | \boldsymbol{\mu}_k, \boldsymbol{\Sigma}_k) \right)$
	\item[Negative Log-likelihood:] The smaller, the better the fit: $-\ln(p(\mathbf{X} | \mathbf{\theta})) = -\Sigma_{n=1}^N \ln\left( \Sigma_{k=1}^K \boldsymbol{\pi}_k \mathcal{N}(\mathbf{x}_n | \boldsymbol{\mu}_k, \boldsymbol{\Sigma}_k) \right)$
\end{compactdesc}

\subsection{AIC / BIC for GMM}
Trade-off: balance between data fit (i.e. likelihood $p(\mathbf{X} | \theta)$) and complexity (i.e. \# of free parameters $\kappa(\cdot)$). Heuristics:
\begin{compactdesc}
	\item[Akaike Information Criterion (AIC):] $\operatorname{AIC}(\theta | \mathbf{X}) = -\ln(p_\theta(\mathbf{X})) + \kappa(\theta)$
	\item[Bayesian Information Criterion (BIC):] $\operatorname{BIC}(\theta | \mathbf{X}) = -\ln(p_\theta(\mathbf{x})) + \frac{1}{2} \kappa(\theta) \ln(N)$
	\item[\# of free params:] (fixed covariance matrix!) $\kappa(\theta) = K \cdot D + T$ ($K$ number of clusters, $D$ is data dimension (i.e. dimension of the $\mu$'s), $T$ number of not fixed $\pi$'s.
\end{compactdesc}
Note that the first term is the log-likelihood multiplied by $-1$. Compare AIC/BIC for different $L$, smaller is better.


\section{Expectation-Maximization (EM) for GMM}
\begin{compactenum}
	\item Initialize $\boldsymbol{\pi}_k^{(0)}, \boldsymbol{\mu}_k^{(0)}, \boldsymbol{\Sigma}_k^{(0)}, t = 1$ for $k = 1, \ldots, K$
	\item E-Step: $q_{k,n} := \mathbb{E}[z_{k,n}] = \frac{\boldsymbol{\pi}_k^{(t-1)} \mathcal{N}(\mathbf{x}_n | \boldsymbol{\mu}_k^{(t-1)}, \boldsymbol{\Sigma}_k^{(t-1)})}{\sum_{j=1}^K \boldsymbol{\pi}_j^{(t-1)} \mathcal{N}(\mathbf{x}_n | \boldsymbol{\mu}_j^{(t-1)}, \boldsymbol{\Sigma}_j^{(t-1)})}$
	\item M-Step: $\boldsymbol{\mu}_k^{(t)} := \frac{\sum_{n=1}^N q_{k,n} \mathbf{x}_n}{\sum_{n=1}^N q_{k,n}}$, $\boldsymbol{\pi}_k^{(t)} := \frac{1}{N} \sum_{n=1}^N q_{k,n}$, and $\Sigma_k^{(t)} = \frac{\sum_{n=1}^N q_{k, n} (\mathbf{x}_n - \boldsymbol{\mu}_k^{(t)})(\mathbf{x}_k - \boldsymbol{\mu}_k^{(t)})^T}{\sum_{n=1}^N q_{k,n}}$
	\item Repeat from (2.) with $t = t + 1$ if not $\| \log p(\mathbf{X} | \boldsymbol{\pi}^{(t)}, \boldsymbol{\mu}^{(t)}, \boldsymbol{\Sigma}^{(t)}) - \log p(\mathbf{X} | \boldsymbol{\pi}^{(t-1)}, \boldsymbol{\mu}^{(t-1)}, \boldsymbol{\Sigma}^{(t-1)}) \| < \epsilon$
\end{compactenum}


\section{Expectation-Maximization (EM) for Inpainting}
\begin{inparaenum}[\color{red} 1.]
	\item Assume initial $\mathbf{z}^{(0)}$, $\mathbf{x}^{(0)}$ original image, $t = 1$
	\item Reconstruct image $\mathbf{x}^{(t)} = \mathbf{M}\mathbf{x}^{(t-1)} + (\mathbf{I} - \mathbf{M})\mathbf{U}\mathbf{z}^{(t-1)}$
	\item Sparse coding of $\mathbf{x}^{(t)}$: $\mathbf{z}^{(t)} \in \argmin_{\mathbf{z}} \|\mathbf{z}\|_0$ s.t. $\|\mathbf{x}^{(t)} - \mathbf{Uz}\|_2 < \sigma$
	\item Repeat (2.) \& (3.) till converging/$\mathit{maxIters}$
	\item Image reconstruction: $\hat{\mathbf{x}} = \mathbf{Mx} + (\mathbf{I} - \mathbf{M})\mathbf{x}^{(t)}$
\end{inparaenum}


\section{Haar Wavelet}
\begin{compactdesc}
	\item[Mother function:] $\psi(t)$ is $1$ for $0 \leq t < \frac{1}{2}$, $-1$ for $\frac{1}{2} \leq t < 1$, $0$ otherwise.
	\item[Haar function:] $\psi_{n,k}(t) = 2^{n/2} \psi(2^n t - k)$, $2^{n/2}$ only needed if normalized values needed!
\end{compactdesc}

\begin{compactitem}
	\item For the not normalized basis: $\mathbf{W}^{-1} = (\mathbf{W^T} \mathbf{W})^{-1} \mathbf{W}^T = \operatorname{diag}(\frac{1}{l_1}, \ldots, \frac{1}{l_n}) \mathbf{W}^T$ where $l_i$ is the norm of the $i$-th column of $\mathbf{W}$. This is valid because $\mathbf{W}$ is orthogonal!
	\item Don't forget $[1, 1, \ldots, 1]^T$: first non-normalized basis!
	\item \textbf{``Coding'' step:} Tranform $\mathbf{x}$ into coordinate vector $\mathbf{c}$
	\item \textbf{``Thresholding'' step:} Set small absolute values in $\mathbf{c}$ to $0$
\end{compactitem}

\section{Optimization}
Gradient of $f: \mathbb{R}^D \to \mathbb{R}$ is $\nabla f(\mathbf{x}) := \left( \frac{\partial f(\mathbf{x})}{\partial \mathbf{x}_1}, \ldots, \frac{\partial f(\mathbf{x})}{\partial \mathbf{x}_D} \right)^T \in \mathbb{R}^D$

\subsection{Convex Functions}
A function $f(x)$ is convex if domain of $f$ (set of input values) is convex (e.g. $\mathbb{R}^D$) and $\forall y, z, \forall t \in [0, 1]: f(t \cdot y + (1-t) \cdot z) \leq t \cdot f(y) + (1-t) \cdot f(z)$.

\subsection{Coordinate Descent}
Update one coordinate (the $d$-th), others are fixed.
\begin{inparaenum}[\color{red} 1.]
	\item Init: $\mathbf{x}^{(0)} \in \mathbb{R}^D$
	\item For $t = 0 \ \text{to} \ \mathit{maxIters}$:
	\item choose $d$ uniformly from $1, \ldots, D$
	\item $u^\star = \argmin_{u \in \mathbb{R}} f(x_1^{(t)}, \ldots, u, \ldots, x_D^{(t)})$
	\item $\mathbf{x}_d^{(t+1)} = u^\star$ and $\mathbf{x}_i^{(t+1)} = \mathbf{x}_i^{(t)}$ for $i \neq d$
\end{inparaenum}

\subsection{Gradient Descent}
\begin{inparaenum}[\color{red} 1.]
	\item Init: $\mathbf{x}^{(0)} \in \mathbb{R}^D$, stepsize usually $\gamma \approx \frac{1}{t}$
	\item For $t = 0 \ \text{to} \ \mathit{maxIters}$: $\mathbf{x}^{(t+1)} = \mathbf{x}^{(t)} - \gamma \nabla f(\mathbf{x}^{(t)})$
\end{inparaenum}

\subsection{Stochastic Gradient Descent}
\begin{inparaenum}[\color{red} 1.]
	\item Init: $\mathbf{x}^{(0)} \in \mathbb{R}^D$, stepsize usually $\gamma \approx \frac{1}{t}$
	\item For $t = 0 \ \text{to} \ \mathit{maxIters}$:
	\item choose $n$ uniformly from $1, \ldots, D$
	\item $\mathbf{x}^{(t+1)} = \mathbf{x}^{(t)} - \gamma \nabla f_n(\mathbf{x}^{(t)})$
\end{inparaenum}

\subsection{Projected Gradient Descent}
In the gradient update step use a projection into constraint space $Q \subseteq \mathbb{R}^D$ with $P_Q(\mathbf{x}) = \argmin_{y \in Q} \|y - x\|$: $\mathbf{x}^{(t+1)} = P_Q\left(\mathbf{x}^{(t)} - \gamma \nabla f(\mathbf{x}^{(t)})\right)$

\subsection{Lagrangian Multipliers}
Minimize  $f(\mathbf{x})$ subject to $g_i(\mathbf{x}) \leq 0,\ i = 1, \ldots, m$ and $h_i(\mathbf{x}) = \mathbf{a}_i^T \mathbf{x} - b_i = 0,\ i = 1, \ldots, p$.
\begin{compactdesc}
	\item[Lagrangian:] $L(\mathbf{x}, \boldsymbol{\lambda}, \boldsymbol{\nu}) := f(\mathbf{x}) + \sum_{i=1}^m \lambda_i g_i(\mathbf{x}) + \sum_{i=1}^p \nu_i h_i(\mathbf{x})$
	\item[Dual function:] $d(\boldsymbol{\lambda}, \boldsymbol{\nu}) := \inf_{\mathbf{x}} L(\mathbf{x}, \boldsymbol{\lambda}, \boldsymbol{\nu}) \in \mathbb{R}$
	\item[Dual Problem:] $\max_{\boldsymbol{\lambda}, \boldsymbol{\nu}} d(\boldsymbol{\lambda}, \boldsymbol{\nu})$ s.t. $\boldsymbol{\lambda} \geq \mathbf{0}$. Result is lower bound of optimum $f(\mathbf{x}^\star)$. Also only interested in $d(\cdot)$ where it is finite (if needed add (in)equality as a ``subject to'').
\end{compactdesc}

\section{Matching Pursuit (MP): Over-Complete Dictionaries}
Sparse approximation of a vector $\mathbf{x}$ onto an over-complete dictionary $\mathcal{D}$ represented by matrix $\mathbf{U}$ using $K$ dictionary entries.
\begin{compactenum}
	\item Initialize: Coefficient vector $\mathbb{R}^{|\mathcal{D}|} \ni \mathbf{z} = \mathbf{0}$, Residual $\mathbf{r} = \mathbf{x}$
	\item While $\|\mathbf{z}\|_0 < K$ repeat the following steps
	\item Find next dictionary index: $d^\star = \argmax_d |\langle \mathbf{u}_d, \mathbf{r} \rangle|$
	\item Update coefficients: $\mathbf{z}_{d^\star} = \mathbf{z}_{d^\star} + \langle \mathbf{u}_{d^\star}, \mathbf{r} \rangle$
	\item Update residual: $\mathbf{r} = \mathbf{r} - \langle \mathbf{u}_{d^\star}, \mathbf{r} \rangle \mathbf{u}_{d^\star}$.
\end{compactenum}

\section{Matching Pursuit (MP): Inpainting}
Diagonal matrix $\mathbf{M}$ with $\mathbf{M}_{i, i} = 1$ if $i$-th pixel known, otherwise $\mathbf{M}_{i,i} = 0$. Interested in $\mathbf{z}^\star \in \argmax_{\mathbf{z}} \|\mathbf{z}\|_0$ subject to $\|\mathbf{M}(\mathbf{x} - \mathbf{Uz})\|_2 < \sigma$. $\mathbf{U}$ over-complete dictionary of known parts, $\sigma$ allowed error. Reconstructed image: $\hat{\mathbf{x}} = \mathbf{Mx} + (\mathbf{I} - \mathbf{M})\mathbf{U} \mathbf{z}^\star$.
\begin{compactenum}
	\item Initialize: Coefficient vector $\mathbf{z} = \mathbf{0}$, residual $\mathbf{r} = \mathbf{M}\mathbf{x}$.
	\item While $\|z\|_0 < K$ and $\|r\|_2 > 0$ repeat the following steps
	\begin{listcols}[2][5pt]
		\item $d^\star = \argmax_d \frac{|\langle \mathbf{M}\mathbf{u}_d, \mathbf{r}\rangle|}{\|\mathbf{M}\mathbf{u}_d\|_2}$
		\item $\mathbf{z}_{d^\star} = \mathbf{z}_{d^\star} + \frac{\langle \mathbf{M}\mathbf{u}_{d^\star}, \mathbf{r} \rangle}{\|\mathbf{M}\mathbf{u}_{d^\star}\|_2^2}$
		\item $\mathbf{r} = \mathbf{r} - \frac{\langle \mathbf{M}\mathbf{u}_{d^\star}, \mathbf{r} \rangle}{\|\mathbf{M}\mathbf{u}_{d^\star}\|_2^2} \mathbf{M}\mathbf{u}_{d^\star}$
	\end{listcols}
\end{compactenum}


\section{Coherence ($m(\cdot)$)}
\begin{compactitem}
	\item $D$ dimension of data, $L$ number of atoms, $L > D$ then over-complete dictionary
	\item $m(\mathbf{U}) = \max_{i,j:\, i \neq j} | \mathbf{u}_i^T \mathbf{u}_j |$
	\item $m(\mathbf{B}) = 0$ if $\mathbf{B}$ is orthogonal matrix and a basis
	\item $m([\mathbf{B}, \mathbf{u}]) \geq \frac{1}{\sqrt{D}}$ if atom $\mathbf{u}$ is added to orthogonal $\mathbf{B}$
\end{compactitem}

\section{Robust PCA}
\begin{compactitem}
	\item Idea: Approximate $\mathbf{X}$ with $\mathbf{L} + \mathbf{S}$, $\mathbf{L}$ is low-rank, $\mathbf{S}$ is sparse.
	\item Minimize $\|\mathbf{L}\|_\star + \lambda \|\mathbf{S}\|_1$ s. t. $\mathbf{L} + \mathbf{S} = \mathbf{X}$
	\item A perfect reconstruction is possible if $\mathbf{S}$ is not low-rank, $\mathbf{L}$ is not sparse, or $\mathbf{X}$ is not low-rank \textit{and} sparse at the same time
\end{compactitem}

\subsection{Alternating Direction Method of Multipliers (ADMM)}
$\min_{\mathbf{x}_1, \mathbf{x}_2} f_1(\mathbf{x}_1) + f_2(\mathbf{x}_2)$ s. t. $\mathbf{A}_1 \mathbf{x}_1 + \mathbf{A}_2 \mathbf{x}_2 = \mathbf{b}$, $f_1, f_2$ convex
\begin{inparaitem}[\color{red}\textbullet]
	\item Augmented Lagrangian: $L_p(\mathbf{x}_1, \mathbf{x}_2, \boldsymbol{\nu}) = f_1(\mathbf{x}_1) + f_2(\mathbf{x}_2) + \boldsymbol{\nu}^T (\mathbf{A}_1 \mathbf{x}_1 + \mathbf{A}_2 \mathbf{x}_2 - \mathbf{b}) + \frac{p}{2}\| \mathbf{A}_1 \mathbf{x}_1 + \mathbf{A}_2 \mathbf{x}_2 - \mathbf{b} \|_2^2$
	\item ADMM: $\mathbf{x}_1^{(t+1)} := \argmin_{\mathbf{x}_1} L_p(\mathbf{x}_1, \mathbf{x}_2^{(t)}, \boldsymbol{\nu}^{(t)})$, $\mathbf{x}_2^{(t+1)} := \argmin_{\mathbf{x}_2} L_p(\mathbf{x}_1^{(t+1)}, \mathbf{x}_2, \boldsymbol{\nu}^{(t)})$, $\boldsymbol{\nu}^{(t+1)} := \boldsymbol{\nu}^{(t)} + p(\mathbf{A}_1 \mathbf{x}_1^{(t+1)} + \mathbf{A}_2 \mathbf{x}_2^{(t+1)} - \mathbf{b})$
	\item ADDM for RPCA: $f_1(\mathbf{L}) = \|\mathbf{L}\|_\star$, $f_2(\mathbf{S}) = \lambda \| \mathbf{S} \|_1$, $\mathbf{A}_1 \mathbf{x}_1 + \mathbf{A}_2 \mathbf{x}_2 = \mathbf{b} \text{ becomes } \mathbf{L} + \mathbf{S} = \mathbf{X}$, therefore $L_p(\mathbf{L}, \mathbf{S}, \boldsymbol{\nu}) = \ldots$, $\boldsymbol{\nu}$ is matrix here!
\end{inparaitem}


\section{Non-Negative Matrix Factorization (NMF)}
\begin{inparaitem}[\color{red}\textbullet]
	\item $\mathbf{X} \in \mathbb{R}^{D \times N}_{\geq 0}$, $D$ words, $N$ documents, $X$ stores word count per document
	\item Target: $\mathbf{X} \approx \mathbf{UZ}$, $\mathbf{U} \in \mathbb{R}^{D \times K}_{\geq 0}$, $\mathbf{Z} \in \mathbb{R}^{K \times N}_{\geq 0}$ (semi-NMF: $\mathbf{U}$ allows negative entries)
\end{inparaitem}

\subsection{LSI}
Approx. $\mathbf{X}$ using SVD keeping $K$ largest singular values with $\tilde{\mathbf{X}} = \mathbf{U} \boldsymbol{\Sigma}_K \mathbf{V}^T$. Interpret $(\mathbf{U} \boldsymbol{\Sigma}_K)^T$ as mapping from $D$ dim. word space to $K$ dim. latent topic space. New document $\mathbf{x} \mapsto (\mathbf{U} \boldsymbol{\Sigma}_K)^T \mathbf{x}$. Compare using inner product $\langle \tilde{\mathbf{x}}_1, \tilde{\mathbf{x}}_2 \rangle = \langle (\mathbf{U} \boldsymbol{\Sigma}_K)^T \mathbf{x}_1, (\mathbf{U} \boldsymbol{\Sigma}_K)^T \mathbf{x}_2 \rangle$: Larger value $\Rightarrow$ more similar.

\subsection{pLSI}
\begin{compactitem}
	\item $P(\mathit{word} | \mathit{doc}) = \sum_{k=1}^K P(\mathit{word} | \mathit{topic} = k) P(\mathit{topic} = k | \mathit{doc})$
	\item $P(\mathit{word}, \mathit{doc}) = P(\mathit{word} | \mathit{doc}) P(\mathit{doc}) \break= \sum_z P(\mathit{word} | \mathit{topic}) P(\mathit{doc} | \mathit{topic}) P(\mathit{topic})$
\end{compactitem}

\textbf{Matrix Factorization View:}
\begin{inparaitem}[\color{red}\textbullet]
	\item Normalize $\mathbf{X}$ resulting in $\hat{\mathbf{X}}$: $\hat{\mathbf{X}}_{d, n} := \frac{\mathbf{X}_{d,n}}{\sum_{d'=1}^D \sum_{n'=1}^N \mathbf{X}_{d', n'}}$
	\item $\hat{\mathbf{X}} \approx \mathbf{U} \mathbf{Z}$, $\hat{\mathbf{X}} \in \mathbb{R}^{D \times N}_{\geq 0}$, $D$ words, $N$ documents
	\item $\mathbf{U}_{d, k} \equiv P(\mathit{word}=d | \mathit{topic} = k)$, $\sum_{d=1}^D \mathbf{U}_{d,k} = 1\, \forall k$
	\item $\mathbf{Z}_{k,n} \equiv P(\mathit{topic} = k | \mathit{doc} = n) P(\mathit{doc} = n)$, $\sum_{k, n} \mathbf{Z}_{k,n} = 1$
	\item $(\mathbf{U}\mathbf{Z})_{d, n} \equiv P(\mathit{word = d}, \mathit{doc} = n)$
	\item Log-likelihood: $\log P(\hat{\mathbf{X}}; \mathbf{U}, \mathbf{Z}) = \sum_{d=1}^D \sum_{n=1}^N \hat{\mathbf{X}}_{d,n} \log \sum_{k=1}^K \mathbf{U}_{d, k} \mathbf{Z}_{k, n}$
\end{inparaitem}

\textbf{Expectation Maximization (EM):}
\begin{compactenum}
	\item E-Step: $P(z|d, w) = \frac{P(z)P(d|z)P(w|z)}{\sum_{z' \in \mathcal{Z}} P(z') P(d|z') P(w|z')}$
	\item M-Steps: $P(w|z) \propto \sum_{d} f(d, w) P(z|d, w)$, $P(d|z) \propto \sum_{w} f(d, w) P(z|d, w)$, $P(z) \propto \sum_{d} \sum_{w} f(d, w) P(z|d, w)$, $f(d, w)$ is frequency of term $w$ in doc $d$.
\end{compactenum}
\begin{inparaitem}[\color{red}\textbullet]
	\item $P(w|z) \in \mathbb{R}^{D \times K}$
	\item $P(d|z) \in \mathbb{R}^{N \times K}$
	\item $P(z) \in \mathbb{R}^{K}$
\end{inparaitem}

\subsection{NMF Algorithm for quadratic cost function}
$\min_{\mathbf{U}, \mathbf{Z}} J(\mathbf{U}, \mathbf{Z}) = \frac{1}{2} \|\mathbf{X} - \mathbf{U}\mathbf{Z}\|_F^2$ s.t. $\mathbf{U}_{d,k} \in [0, \infty)\ \forall d, k$ \&  $\mathbf{Z}_{k, n} \in [0, \infty)\ \forall k, n$
\begin{inparaenum}[\color{red} 1.]
	\item Init: $\mathbf{U} \gets \mathrm{rand}(D,K), \mathbf{Z} \gets \mathrm{rand}(K,N)$
	\item repeat for $\mathit{maxIters}$:
	\item Update $\mathbf{U}$: $u_{d,k} \gets u_{d,k} \frac{(\mathbf{X}\mathbf{Z}^T)_{d,k}}{(\mathbf{U}\mathbf{Z}\mathbf{Z}^T)_{d,k}}$
	\item Update $\mathbf{Z}$: $z_{k,n} \gets z_{k,n} \frac{(\mathbf{U}^T\mathbf{X})_{k,n}}{(\mathbf{U}^T\mathbf{U}\mathbf{Z})_{k,n}}$
\end{inparaenum}

\raggedcolumns
\end{multicols*}
\end{document}
